% Türkçe Tez Şablonu - Yazım Kurallarına Uygun
% Turkish Thesis Template - Compliant with Writing Rules

\documentclass[12pt,a4paper,oneside]{report}

% Minimal paketler - sadece gerekli olanlar
\usepackage[utf8]{inputenc}
\usepackage[T1]{fontenc}
\usepackage[left=4cm, right=2.5cm, top=3cm, bottom=2.5cm]{geometry}
\usepackage{setspace}
\usepackage{graphicx}

% ===============================
% TEZ BİLGİLERİ
% ===============================
\newcommand{\thesistitle}{ {{thesisTitle}} }
\newcommand{\authorname}{ {{authorName}} }
\newcommand{\advisorname}{ {{advisorName}} }
\newcommand{\university}{ {{university}} }
\newcommand{\faculty}{ {{faculty}} }
\newcommand{\department}{ {{department}} }
\newcommand{\submissiondate}{ {{submissionDate}} }

% ===============================
% SAYFA NUMARALAMA VE FORMATLAR
% ===============================
\pagestyle{plain}

% Ana metin için 1.5 satır aralığı
\onehalfspacing

% Başlık formatları - standart LaTeX ile
\makeatletter
% Bölüm başlıkları için ortalama ve 5cm üstten başlama
\renewcommand{\@makechapterhead}[1]{%
  \vspace*{2cm}% 3cm üst kenar + 2cm = 5cm
  {\parindent \z@ \centering \normalfont
    \ifnum \c@secnumdepth >\m@ne
        \large\bfseries \@chapapp\space \thechapter
        \par\nobreak
        \vskip 20\p@
    \fi
    \interlinepenalty\@M
    \large \bfseries #1\par\nobreak
    \vskip 20\p@
  }}

\renewcommand{\@makeschapterhead}[1]{%
  \vspace*{2cm}%
  {\parindent \z@ \centering
    \normalfont
    \interlinepenalty\@M
    \large \bfseries #1\par\nobreak
    \vskip 20\p@
  }}

% Section formatları
\renewcommand{\section}{\@startsection{section}{1}{\z@}%
  {-3.5ex \@plus -1ex \@minus -.2ex}%
  {2.3ex \@plus.2ex}%
  {\normalfont\normalsize\bfseries}}

\renewcommand{\subsection}{\@startsection{subsection}{2}{\z@}%
  {-3.25ex\@plus -1ex \@minus -.2ex}%
  {1.5ex \@plus .2ex}%
  {\normalfont\normalsize\bfseries}}
\makeatother

% Tablo ve şekil numaralandırma (Bölüm.Sıra)
\renewcommand{\thetable}{\thechapter.\arabic{table}}
\renewcommand{\thefigure}{\thechapter.\arabic{figure}}
\renewcommand{\theequation}{\thechapter.\arabic{equation}}

% İçindekiler başlıkları - manuel
\renewcommand{\contentsname}{İÇİNDEKİLER}
\renewcommand{\listfigurename}{ŞEKİLLER DİZİNİ}
\renewcommand{\listtablename}{TABLOLAR DİZİNİ}
\renewcommand{\bibname}{KAYNAKLAR}
\renewcommand{\appendixname}{EK}
\renewcommand{\chaptername}{}

% ===============================
% BELGE BAŞLANGICI
% ===============================
\begin{document}

% ===============================
% DIŞ KAPAK
% ===============================
\begin{titlepage}
\centering
\vspace*{2cm}

{\fontsize{14}{21}\selectfont\bfseries \university \par}
\vspace{0.5cm}
{\fontsize{14}{21}\selectfont\bfseries \faculty \par}
\vspace{0.5cm}
{\fontsize{14}{21}\selectfont\bfseries \department \par}

\vspace{3cm}

{\fontsize{16}{24}\selectfont\bfseries \thesistitle \par}

\vspace{3cm}

{\fontsize{14}{21}\selectfont LİSANS TEZİ \par}

\vspace{2cm}

{\fontsize{14}{21}\selectfont \authorname \par}

\vspace{2cm}

{\fontsize{12}{18}\selectfont Danışman: \advisorname \par}

\vfill

{\fontsize{12}{18}\selectfont \submissiondate \par}

\end{titlepage}

% ===============================
% İÇ KAPAK
% ===============================
\begin{titlepage}
\centering
\vspace*{2cm}

{\fontsize{14}{21}\selectfont\bfseries \university \par}
\vspace{0.5cm}
{\fontsize{14}{21}\selectfont\bfseries \faculty \par}
\vspace{0.5cm}
{\fontsize{14}{21}\selectfont\bfseries \department \par}

\vspace{3cm}

{\fontsize{16}{24}\selectfont\bfseries \thesistitle \par}

\vspace{3cm}

{\fontsize{14}{21}\selectfont LİSANS TEZİ \par}

\vspace{2cm}

{\fontsize{14}{21}\selectfont \authorname \par}

\vspace{2cm}

{\fontsize{12}{18}\selectfont Danışman: \advisorname \par}

\vfill

{\fontsize{12}{18}\selectfont \submissiondate \par}

\end{titlepage}

% ===============================
% BAŞLANGIÇ KISMI - ROMEN RAKAMLARI
% ===============================
\pagenumbering{roman}

% ===============================
% ÖNSÖZ
% ===============================
\chapter*{ÖNSÖZ}
\addcontentsline{toc}{chapter}{ÖNSÖZ}

\singlespacing % Önsöz için tek satır aralığı

Bu tez çalışmasında {{thesisTitle}} konusu ele alınmıştır. Çalışma boyunca modern teknolojilerin analizi ve uygulanması hedeflenmiştir.

Tez çalışmam boyunca bana yol gösteren değerli danışmanım {{advisorName}}'a, desteklerini hiç esirgemeyen aileme ve arkadaşlarıma teşekkürlerimi sunarım.

\vspace{2cm}
\hfill {{authorName}} \\
\hfill {{submissionDate}}

\onehalfspacing % Ana metne dönüş için 1.5 aralık

\clearpage

% ===============================
% İÇİNDEKİLER
% ===============================
\tableofcontents
\clearpage

% ===============================
% ŞEKİLLER DİZİNİ
% ===============================
\listoffigures
\clearpage

% ===============================
% TABLOLAR DİZİNİ
% ===============================
\listoftables
\clearpage

% ===============================
% SİMGELER VE KISALTMALAR
% ===============================
\chapter*{SİMGELER}
\addcontentsline{toc}{chapter}{SİMGELER}

\singlespacing % Simgeler için tek satır aralığı

\section*{Simgeler}
\begin{tabular}{@{}ll@{}}
$A$ & Alan, m² \\
$T$ & Sıcaklık, K \\
$v$ & Hız, m/s \\
$\rho$ & Yoğunluk, kg/m³ \\
$\alpha$ & Doğruluk oranı \\
$\beta$ & Güven aralığı \\
\end{tabular}

\section*{Kısaltmalar}
\begin{tabular}{@{}ll@{}}
A.I. & Artificial Intelligence \\
M.L. & Machine Learning \\
S.V.M. & Support Vector Machine \\
R.F. & Random Forest \\
N.N. & Neural Network \\
\end{tabular}

\onehalfspacing
\clearpage

% ===============================
% TÜRKÇE ÖZET
% ===============================
\chapter*{ {{thesisTitle}} }
\addcontentsline{toc}{chapter}{ÖZET}

\singlespacing % Özet için tek satır aralığı

\section*{ÖZET}

{{abstractTr}}

\textbf{Anahtar Kelimeler:} {{keywordsTr}}

\onehalfspacing
\clearpage

% ===============================
% İNGİLİZCE ÖZET
% ===============================
\chapter*{ {{thesisTitleEn}} }
\addcontentsline{toc}{chapter}{ABSTRACT}

\singlespacing % Abstract için tek satır aralığı

\section*{ABSTRACT}

{{abstractEn}}

\textbf{Keywords:} {{keywordsEn}}

\onehalfspacing
\clearpage

% ===============================
% ANA METİN - ARAP RAKAMLARI
% ===============================
\pagenumbering{arabic}

{{#chapters}}
\chapter{ {{title}} }

{{#sections}}
\section{ {{title}} }

{{content}}

{{/sections}}

{{#isChapter3}}
% Malzeme ve Yöntem bölümü için tablo örneği

\begin{table}[h!]
\centering
\caption{Veri seti özellikleri}
\label{tab:dataset}
\begin{tabular}{|l|c|}
\hline
\textbf{Özellik} & \textbf{Değer} \\
\hline
Toplam Örnek Sayısı & 10,000 \\
Özellik Sayısı & 50 \\
Eğitim Seti (\%) & 70 \\
Test Seti (\%) & 20 \\
Doğrulama Seti (\%) & 10 \\
\hline
\end{tabular}
\end{table}

Tablo \ref{tab:dataset}'de veri seti özellikleri detaylandırılmıştır.

\begin{figure}[h!]
\centering
\rule{10cm}{6cm} % Şekil yerine geçici kutu
\caption{Hibrit model mimarisi}
\label{fig:model_architecture}
\end{figure}

Şekil \ref{fig:model_architecture}'de önerilen hibrit modelin genel mimarisi gösterilmektedir.
{{/isChapter3}}

{{#isChapter4}}
% Bulgular bölümü için tablo ve şekil örnekleri

\begin{table}[h!]
\centering
\caption{Algoritma performans karşılaştırması}
\label{tab:performance}
\begin{tabular}{|l|c|c|c|}
\hline
\textbf{Algoritma} & \textbf{Doğruluk (\%)} & \textbf{Kesinlik (\%)} & \textbf{Duyarlılık (\%)} \\
\hline
Random Forest & 78.5 & 76.2 & 80.1 \\
SVM & 82.3 & 81.7 & 83.0 \\
Neural Network & 85.1 & 84.3 & 85.9 \\
Hibrit Model & \textbf{87.2} & \textbf{86.8} & \textbf{87.6} \\
\hline
\end{tabular}
\end{table}

Tablo \ref{tab:performance}'de görüldüğü gibi, önerilen hibrit model en yüksek performansı göstermiştir.

\begin{figure}[h!]
\centering
\rule{10cm}{6cm} % Şekil yerine geçici kutu
\caption{Algoritmaların ROC eğrisi karşılaştırması}
\label{fig:roc_curve}
\end{figure}

Şekil \ref{fig:roc_curve}'de farklı algoritmaların ROC eğrileri karşılaştırılmıştır.

\begin{table}[h!]
\centering
\caption{Karışıklık matrisi - Hibrit Model}
\label{tab:confusion_matrix}
\begin{tabular}{|c|c|c|}
\hline
 & \textbf{Pozitif Tahmin} & \textbf{Negatif Tahmin} \\
\hline
\textbf{Pozitif Gerçek} & 1,562 & 218 \\
\textbf{Negatif Gerçek} & 156 & 1,064 \\
\hline
\end{tabular}
\end{table}

Denklem (\ref{eq:accuracy}) ile doğruluk oranı hesaplanmıştır:

\begin{equation}
Accuracy = \frac{TP + TN}{TP + TN + FP + FN}
\label{eq:accuracy}
\end{equation}

burada TP (True Positive), TN (True Negative), FP (False Positive) ve FN (False Negative) değerlerini temsil eder.
{{/isChapter4}}

{{/chapters}}

% ===============================
% KAYNAKLAR
% ===============================
\chapter*{KAYNAKLAR}
\addcontentsline{toc}{chapter}{KAYNAKLAR}

\singlespacing % Kaynaklar için tek satır aralığı

1. Wechsatol, W., Lorente, S., Bejan, A., ``Tree-shaped insulated design for uniform distribution of hot water over an area'', \textit{Int. J. Heat Mass Transfer}, 44, 3111-3123, (2001).

2. Özkan, M., Yılmaz, S., ``Hibrit Yaklaşımların Performans Analizi'', \textit{Bilgisayar Bilimleri Dergisi}, 12(2), 67-89, (2022).

3. Chen, L., Wang, X., ``Deep Learning Applications in Data Processing'', \textit{IEEE Transactions on Neural Networks}, 34(7), 234-251, (2023).

4. Anderson, R., Thompson, P., ``Comparative Study of Classification Algorithms'', \textit{Machine Learning Review}, 28(4), 445-467, (2022).

5. Garcia, M., Rodriguez, C., ``Feature Selection Methods in AI Systems'', \textit{Artificial Intelligence Quarterly}, 15(1), 78-92, (2023).

6. Kumar, S., Patel, N., ``Performance Evaluation Metrics for ML Models'', \textit{Data Science Journal}, 8(3), 156-174, (2022).

7. Williams, D., Lee, H., ``Ensemble Methods in Machine Learning'', \textit{Computer Science Today}, 19(6), 301-318, (2023).

8. Taylor, B., Wilson, M., ``Statistical Analysis of AI Performance'', \textit{Statistics and Computing}, 41(2), 89-108, (2022).

\onehalfspacing

% ===============================
% EKLER
% ===============================
\appendix
\renewcommand{\thechapter}{\Alph{chapter}}
\renewcommand{\thetable}{\Alph{chapter}.\arabic{table}}
\renewcommand{\thefigure}{\Alph{chapter}.\arabic{figure}}

\chapter{DENEYSEL SONUÇLAR}

\singlespacing % Ekler için tek satır aralığı

Bu ekte, tez çalışmasında kullanılan detaylı deneysel sonuçlar ve istatistiksel analizler yer almaktadır.

\section{Detaylı Performans Metrikleri}

\begin{table}[h!]
\centering
\caption{Algoritmaların detaylı performans sonuçları}
\begin{tabular}{|l|c|c|c|c|}
\hline
\textbf{Algoritma} & \textbf{F1-Score} & \textbf{AUC} & \textbf{Kappa} & \textbf{MCC} \\
\hline
Random Forest & 0.782 & 0.856 & 0.741 & 0.746 \\
SVM & 0.824 & 0.889 & 0.798 & 0.801 \\
Neural Network & 0.853 & 0.921 & 0.832 & 0.835 \\
Hibrit Model & \textbf{0.872} & \textbf{0.945} & \textbf{0.861} & \textbf{0.864} \\
\hline
\end{tabular}
\end{table}

\onehalfspacing

% ===============================
% ÖZGEÇMİŞ
% ===============================
\chapter*{ÖZGEÇMİŞ}
\addcontentsline{toc}{chapter}{ÖZGEÇMİŞ}

\singlespacing % Özgeçmiş için tek satır aralığı

\textbf{Kişisel Bilgiler:}

\begin{tabular}{@{}ll@{}}
Ad Soyad: & {{authorName}} \\
Doğum Yeri ve Tarihi: & {{personalInfo.birthPlace}}, {{personalInfo.birthDate}} \\
E-posta: & {{personalInfo.email}} \\
\end{tabular}

\vspace{0.5cm}

\textbf{Eğitim:}

\begin{tabular}{@{}ll@{}}
{{#personalInfo.education}}
{{year}} & {{level}}, {{department}}, {{school}} \\
{{/personalInfo.education}}
\end{tabular}

\vspace{0.5cm}

\textbf{İş Deneyimi:}

\begin{tabular}{@{}ll@{}}
{{#personalInfo.experience}}
{{year}} & {{position}}, {{company}} \\
{{/personalInfo.experience}}
\end{tabular}

\vspace{0.5cm}

\textbf{Yabancı Dil:}

İngilizce (İyi), Almanca (Orta)

\textbf{Yayınlar:}

1. {{authorName}}, {{advisorName}}, "{{thesisTitle}}: Bir Uygulama", \textit{Ulusal Bilgisayar Bilimleri Kongresi}, İstanbul, 2024.

\end{document}